% Template for PLoS
% Version 3.5 March 2018
%
% % % % % % % % % % % % % % % % % % % % % %
%
% -- IMPORTANT NOTE
%
% This template contains comments intended
% to minimize problems and delays during our production
% process. Please follow the template instructions
% whenever possible.
%
% % % % % % % % % % % % % % % % % % % % % % %
%
% Once your paper is accepted for publication,
% PLEASE REMOVE ALL TRACKED CHANGES in this file
% and leave only the final text of your manuscript.
% PLOS recommends the use of latexdiff to track changes during review, as this will help to maintain a clean tex file.
% Visit https://www.ctan.org/pkg/latexdiff?lang=en for info or contact us at latex@plos.org.
%
%
% There are no restrictions on package use within the LaTeX files except that
% no packages listed in the template may be deleted.
%
% Please do not include colors or graphics in the text.
%
% The manuscript LaTeX source should be contained within a single file (do not use \input, \externaldocument, or similar commands).
%
% % % % % % % % % % % % % % % % % % % % % % %
%
% -- FIGURES AND TABLES
%
% Please include tables/figure captions directly after the paragraph where they are first cited in the text.
%
% DO NOT INCLUDE GRAPHICS IN YOUR MANUSCRIPT
% - Figures should be uploaded separately from your manuscript file.
% - Figures generated using LaTeX should be extracted and removed from the PDF before submission.
% - Figures containing multiple panels/subfigures must be combined into one image file before submission.
% For figure citations, please use "Fig" instead of "Figure".
% See http://journals.plos.org/plosone/s/figures for PLOS figure guidelines.
%
% Tables should be cell-based and may not contain:
% - spacing/line breaks within cells to alter layout or alignment
% - do not nest tabular environments (no tabular environments within tabular environments)
% - no graphics or colored text (cell background color/shading OK)
% See http://journals.plos.org/plosone/s/tables for table guidelines.
%
% For tables that exceed the width of the text column, use the adjustwidth environment as illustrated in the example table in text below.
%
% % % % % % % % % % % % % % % % % % % % % % % %
%
% -- EQUATIONS, MATH SYMBOLS, SUBSCRIPTS, AND SUPERSCRIPTS
%
% IMPORTANT
% Below are a few tips to help format your equations and other special characters according to our specifications. For more tips to help reduce the possibility of formatting errors during conversion, please see our LaTeX guidelines at http://journals.plos.org/plosone/s/latex
%
% For inline equations, please be sure to include all portions of an equation in the math environment.  For example, x$^2$ is incorrect; this should be formatted as $x^2$ (or $\mathrm{x}^2$ if the romanized font is desired).
%
% Do not include text that is not math in the math environment. For example, CO2 should be written as CO\textsubscript{2} instead of CO$_2$.
%
% Please add line breaks to long display equations when possible in order to fit size of the column.
%
% For inline equations, please do not include punctuation (commas, etc) within the math environment unless this is part of the equation.
%
% When adding superscript or subscripts outside of brackets/braces, please group using {}.  For example, change "[U(D,E,\gamma)]^2" to "{[U(D,E,\gamma)]}^2".
%
% Do not use \cal for caligraphic font.  Instead, use \mathcal{}
%
% % % % % % % % % % % % % % % % % % % % % % % %
%
% Please contact latex@plos.org with any questions.
%
% % % % % % % % % % % % % % % % % % % % % % % %

\documentclass[10pt,letterpaper]{article}
\usepackage[top=0.85in,left=1.625in,right=1.625in,footskip=0.75in]{geometry}

% amsmath and amssymb packages, useful for mathematical formulas and symbols
\usepackage{amsmath,amssymb}

% Use adjustwidth environment to exceed column width (see example table in text)
\usepackage{changepage}

% Use Unicode characters when possible
\usepackage[utf8x]{inputenc}

% textcomp package and marvosym package for additional characters
\usepackage{textcomp,marvosym}

% cite package, to clean up citations in the main text. Do not remove.
\usepackage{cite}

% Use nameref to cite supporting information files (see Supporting Information section for more info)
% TODO: hidelinks.
\usepackage{nameref}
\usepackage[hidelinks]{hyperref}

% line numbers
\usepackage[right]{lineno}

% ligatures disabled
\usepackage{microtype}
\DisableLigatures[f]{encoding = *, family = * }

% color can be used to apply background shading to table cells only
\usepackage[table]{xcolor}

% array package and thick rules for tables
\usepackage{array}

% create "+" rule type for thick vertical lines
\newcolumntype{+}{!{\vrule width 2pt}}

% create \thickcline for thick horizontal lines of variable length
\newlength\savedwidth
\newcommand\thickcline[1]{%
  \noalign{\global\savedwidth\arrayrulewidth\global\arrayrulewidth 2pt}%
  \cline{#1}%
  \noalign{\vskip\arrayrulewidth}%
  \noalign{\global\arrayrulewidth\savedwidth}%
}

% \thickhline command for thick horizontal lines that span the table
\newcommand\thickhline{\noalign{\global\savedwidth\arrayrulewidth\global\arrayrulewidth 2pt}%
\hline
\noalign{\global\arrayrulewidth\savedwidth}}


% Remove comment for double spacing
%\usepackage{setspace}
%\doublespacing

% Text layout
%\raggedright
\setlength{\parindent}{0.5cm}
\textwidth 5.25in
\textheight 8.75in

% Bold the 'Figure #' in the caption and separate it from the title/caption with a period
% Captions will be left justified
\usepackage[aboveskip=1pt,labelfont=bf,labelsep=period,justification=raggedright,singlelinecheck=off]{caption}
\renewcommand{\figurename}{Figure}

% Use the PLoS provided BiBTeX style
\bibliographystyle{plos2015}

% Remove brackets from numbering in List of References
\makeatletter
\renewcommand{\@biblabel}[1]{\quad#1.}
\makeatother



% Header and Footer with logo
\usepackage{lastpage,fancyhdr,graphicx}
\usepackage{epstopdf}
%\pagestyle{myheadings}
\pagestyle{fancy}
\fancyhf{}
%\setlength{\headheight}{27.023pt}
%\lhead{\includegraphics[width=2.0in]{PLOS-submission.eps}}
\rfoot{\thepage/\pageref{LastPage}}
\renewcommand{\headrulewidth}{0pt}
\renewcommand{\footrule}{\hrule height 2pt \vspace{2mm}}
% \fancyheadoffset[L]{2.25in}
% \fancyfootoffset[L]{2.25in}
% \lfoot{\today}

%% Include all macros below

\newcommand{\ocp}{
\begin{adjustwidth}{-1.125in}{-1.125in} % Comment out/remove adjustwidth environment if table fits in text column.
% \mathrlap causes the contents to not take up horizontal space, allowing
% overlapping columns.
\begin{align}
    \begin{aligned}
        \mbox{minimize}
         \quad & \sum_j w_{j} J_{j}(t_0, t_f, y_0, y_f, x_{0}, \mathrlap{x_{f}, \lambda_0, \lambda_f, p, S_{c,j})} && \textrm{costs} \\
        & \quad\quad S_{c,j} = \int_{t_0}^{t_f} s_{c,j}(t, y, x, \lambda, p)\,dt  \\
        \mbox{subject to}
         \quad & \dot{q} = u \\
         & M(q, p)\dot{u} + G(q, p)^T \lambda = f_{\mathrm{app}}(t, y, x, p) - f_{\mathrm{inertial}}(q, u, p)  && \textrm{multibody dynamics} \\
         & \dot{z}_\textrm{ex}(t) = f_{\dot{z},\textrm{ex}}(t, y, x, \lambda, p) && \textrm{auxiliary dynamics, explicit} \\
         & 0 = f_{\dot{z},\textrm{im}}(t, y, \dot{z}_{\textrm{im}}, x, \lambda, p) &&  \textrm{auxiliary dynamics, implicit}\\
         & 0 = \phi(q, p) && \textrm{kinematic constraints} \\
         & V_{L,k} \leq V_k(t_0, t_f, y_0, y_f, x_{0}, x_{f}, \lambda_0, \lambda_f, p, S_{b,k}) \leq V_{U,k} && \textrm{boundary constraints} \\
         & \quad\quad S_{b,k} = \int_{t_0}^{t_f} s_{b,k}(t, y, x, \lambda, p)\,dt \quad k = 1, \ldots, K\\
         & g_{L} \leq g(t, y, x, \lambda, p) \leq g_{U} && \textrm{path constraints} \\
         & y_{0,L} \leq y_0 \leq y_{0,U} \quad\quad y_{f,L} \leq y_f \leq y_{f,U} && \textrm{initial and final states} \\
         & x_{0,L} \leq x_0 \leq x_{0,U} \quad\quad x_{f,L} \leq x_f \leq x_{f,U} && \textrm{initial and final controls} \\
         \mbox{with respect to} \quad
         & t_0 \in [t_{0,L}, t_{0,U}] && \textrm{initial time} \\
         & t_f \in [t_{f,L}, t_{f,U}] && \textrm{final time} \\
         & y(t) = (q(t), u(t), z(t)) \in [y_{L}, y_{U}] && \textrm{states} \\
         & x(t) \in [x_{L}, x_{U}] && \textrm{controls} \\
         & \lambda(t) && \textrm{Lagrange multipliers} \\
         & p \in [p_{L}, p_{U}] && \textrm{time-invariant parameters}
    \end{aligned}
    \label{ocp}
\end{align}
\end{adjustwidth}
}

\newcommand{\prescribed}{
\begin{adjustwidth}{-1.125in}{-1.125in} % Comment out/remove adjustwidth environment if table fits in text column.
\begin{align}
    \begin{aligned}
        \mbox{minimize} \quad & \mathrlap{\sum_j w_j J_{j}(t_0, t_f, \hat{q}_0, \hat{q}_f, \hat{u}_0, \hat{u}_f, z_0, z_f, x_{0}, x_{f}, \lambda_0, \lambda_f, p, S_{c,j})} &&&\hspace{6em}& \textrm{costs} \\
        & \quad\quad \mathrlap{S_{c,j} = \int_{t_0}^{t_f} s_{c,j}(t, \hat{q}, \hat{u}, z, x, \lambda, p)~dt} \\
        \mbox{subject to} \quad &
         M(\hat{q}, p)\hat{\dot{u}} + \mathrlap{G(\hat{q}, p)^T \lambda = f_{\textrm{app}}(t, \hat{q}, \hat{u}, z, x, p) - f_{\textrm{inertial}}(\hat{q}, \hat{u}, p)} &&&& \textrm{multibody dynamics} \\
        & \mathrlap{\dot{z}_{\textrm{ex}}(t) = f_{\dot{z},\textrm{ex}}(t, \hat{q}, \hat{u}, z, x, \lambda, p)} &&&& \textrm{auxiliary dynamics, explicit} \\
        & \mathrlap{0 = f_{\dot{z},\textrm{im}}(t, \hat{q}, \hat{u}, z, \dot{z}_\textrm{im}, x, \lambda, p)} &&&& \textrm{auxiliary dynamics, implicit}\\
        & V_{L,k} \leq V_k(t_0, t_f, \hat{q}_0, \mathrlap{\hat{q}_f, \hat{u}_0, \hat{u}_f, z_0, z_f, x_{0}, x_{f}, \lambda_0, \lambda_f, p, S_{b,k}) \leq V_{U,k}}  &&&& \textrm{boundary constraints} \\
        & \quad\quad \mathrlap{S_{b,k} = \int_{t_0}^{t_f} s_{b,k}(t, \hat{q}, \hat{u}, z, x, \lambda, p)~dt \quad k = 1, \ldots, K} \\
        & \mathrlap{g_{L} \leq g(t, \hat{q}, \hat{u}, z, x, \lambda, p) \leq g_{U}} &&&& \textrm{path constraints} \\
        & z_{0,L} \leq z_0 \leq z_{0,U} && z_{f,L} \leq z_f \leq z_{f,U} && \textrm{initial and final states} \\
        & x_{0,L} \leq x_0 \leq x_{0,U} && x_{f,L} \leq x_f \leq x_{f,U} && \textrm{initial and final controls} \\
        \mbox{with respect to} \quad
        & t_0 \in [t_{0,L}, t_{0,U}] && \textrm{initial time} \\
        & t_f \in [t_{f,L}, t_{f,U}] && \textrm{final time} \\
        & z(t) \in [z_{L}, z_{U}] && \textrm{auxiliary states} \\
        & x(t) \in [x_{L}, x_{U}] && \textrm{controls} \\
        & \lambda(t) && \textrm{Lagrange multipliers} \\
        & p \in [p_{L}, p_{U}]. && \textrm{time-invariant parameters}
    \end{aligned}
\end{align}
\end{adjustwidth}
}

\newcommand{\traptau}{
\begin{equation}
    \begin{gathered}
        0 = \tau_0 < \tau_1 < \tau_2 < \ldots < \tau_i < \ldots < \tau_{n - 1} < \tau_n = 1, \\
        \begin{aligned}
        t_i &= (t_f - t_0) \tau_i + t_0, \\
        h_i &= (t_f - t_0)(\tau_i - \tau_{i-1}).
        \end{aligned}
    \end{gathered}
\end{equation}
}

\newcommand{\trapfunc}{
\begin{equation}
    \textrm{trap}_i(F(\eta, p)) = \frac{1}{2} h_i (F(\eta_{i-1}, p) + F(\eta_i, p)),
\end{equation}
}

\newcommand{\explicitmultibody}{
\begin{equation}
    f_{\dot{u}}(t, y, x, \lambda, p) =
    M(q, p)^{-1}\big[(f_{\textrm{app}}(t, y, x, p) - f_{\textrm{inertial}}(q, u, p) - G(q, p)^T \lambda\big].
\end{equation}
}

\newcommand{\trapnlp}{
\begin{align}
    \begin{aligned}
        \mbox{minimize} \quad
         & \sum_j w_j J_{j}(t_0, t_f, \mathrlap{y_0, y_n, x_{0}, x_{n}, \lambda_0, \lambda_n, p, S_{c,j})
          + w_{\lambda} \sum_{i=1}^{n} \textrm{trap}_i(\|\lambda\|_2^2)}  \\
         & \quad\quad \mathrlap{S_{c,j} = \sum_{i=1}^{n} \textrm{trap}_i(s_{c,j}(t, y, x, \lambda, p))} \\
        \mbox{subject to} \quad
         & \mathrlap{q_i = q_{i-1} + \textrm{trap}_i(u)} &&&& i = 1, \ldots, n \\
         & \mathrlap{u_i = u_{i-1} + \textrm{trap}_i(f_{\dot{u}}(t, y, x, \lambda, p))}  &&&& i = 1, \ldots, n \\
         & \mathrlap{z_{\textrm{ex},i} = z_{\textrm{ex},i-1} + \textrm{trap}_i(f_{\dot{z},\textrm{ex}}(t, y, x, \lambda, p))} &&&& i = 1, \ldots, n \\
         & \mathrlap{z_{\textrm{im},i} = z_{\textrm{im},i-1} + \textrm{trap}_i(\zeta)} &&&& i = 1, \ldots, n \\
         & \mathrlap{0 = f_{\dot{z},\textrm{im}}(t_i, y_i, \zeta_i, x_i, \lambda_i, p)} &&&& i = 0, \ldots, n \\
         & 0 = \phi(q_i, p)  &&&& i = 0, \ldots, n\\
         & V_{L,k} \leq V_k(t_0, t_f, y_0, \mathrlap{y_f, x_{0}, x_{f}, \lambda_0, \lambda_f, p, S_{b,k}) \leq V_{U,k}} \\
         & \quad\quad \mathrlap{S_{b,k} = \sum_{i=1}^{n} \textrm{trap}_i(s_{b,k}(t, y, x, \lambda, p))} &&&& k = 1, \ldots, K \\
         & \mathrlap{g_{L} \leq g(t_i, y_i, x_{i}, \lambda_i, p) \leq g_{U}}  &&&& i = 0, \ldots, n\\
         \mbox{with respect to} \quad
         & t_0 \in [t_{0,L}, t_{0,U}] && t_n \in [t_{f,L}, t_{f,U}] \\
         & y_0 \in [y_{0,L}, y_{0,U}] && y_n \in [y_{f,L}, y_{f,U}] \\
         & y_i \in [y_{L}, y_{U}] &&&& i = 1, \ldots, n - 1\\
         & \zeta_i \in [\zeta_{L}, \zeta_{U}] &&&& i = 0, \ldots, n \\
         & x_0 \in [x_{0,L}, x_{0,U}] && x_n \in [x_{f,L}, x_{f,U}] \\
         & x_i \in [x_{L}, x_{U}] &&&& i = 1, \ldots, n - 1\\
         & \lambda_i \in [\lambda_L, \lambda_U] &&&& i = 0, \ldots, n \\
         & p \in [p_{L}, p_{U}].
    \end{aligned}
    \label{trapnlp}
\end{align}
}

\newcommand{\trapimplicit}{
\begin{align}
    \begin{aligned}
    \mbox{subject to} \quad
         & u_i = u_{i-1} + \textrm{trap}_i(\upsilon)  && i = 1, \ldots, n \\
         & M(q_i, p)\upsilon_i + G(q_i, p)^T \lambda_i =
          f_{\textrm{app},i} - f_{\textrm{inertial},i} && i = 0, \ldots, n \\
    \mbox{with respect to} \quad
         & \upsilon_i \in [-\upsilon_{B}, \upsilon_{B}] && i = 0, \ldots, n.\\
    \end{aligned}
\end{align}
}

\newcommand{\hermitesimpsontau}{
\begin{equation}
    \begin{gathered}
        0 = \tau_0 < \tau_1 < \tau_2 < \ldots < \tau_i < \ldots < \tau_{n - 1} < \tau_n = 1, \\
        \begin{aligned}
        \bar{\tau}_i &= 0.5 (\tau_{i-1} + \tau_i), \\
        t_i &= (t_f - t_0) \tau_i + t_0, \\
        \bar{t}_i &= (t_f - t_0) \bar{\tau}_i + t_0, \\
        h_i &= (t_f - t_0)(\tau_i - \tau_{i-1}),
        \end{aligned}
    \end{gathered}
\end{equation}
}

\newcommand{\hermitesimpsonfuncs}{
\begin{align}
    \textrm{hermite}_i(\eta, F(\eta, p)) &= \frac{1}{2} (\eta_{i-1} + \eta_i) + \frac{h_i}{8} \big(F(\eta_{i-1}, p) - F(\eta_i, p)\big), \\
    \textrm{simpson}_i(F(\eta, p)) &= \frac{h_i}{6} \big(F(\eta_{i-1}, p) + 4 F(\bar{\eta}_i, p) + F(\eta_i, p)\big),
\end{align}
}

\newcommand{\hermitesimpsonnlp}{
\begin{align}
    \begin{aligned}
        \mbox{minimize} \quad
         & \sum_j w_j J_{c,j}(t_0, t_f, \mathrlap{y_0, y_n, x_{0}, x_{n}, \lambda_0, \lambda_n, p, S_{c,j})
         + w_{\lambda} \sum_{i=1}^{n} \textrm{simpson}_i(\|\lambda\|_2^2)}  \\
         & \quad\quad \mathrlap{S_{c,j} = \sum_{i=1}^{n} \textrm{simpson}_i(s_{c,j}(t, y, x, \lambda, p))} \\
        \mbox{subject to} \quad
         & \mathrlap{\bar{q}_i = \textrm{hermite}_i(q, f_{\dot{q}}(q, u, \gamma, p))} &&&& i = 1, \ldots, n \\
         & \mathrlap{q_i = q_{i-1} + \textrm{simpson}_i(f_{\dot{q}}(q, u, \gamma, p))} &&&& i = 1, \ldots, n \\
         & \mathrlap{\bar{u}_i = \textrm{hermite}_i(u, f_{\dot{u}}(t, y, x, \lambda, p))} &&&& i = 1, \ldots, n \\
         & \mathrlap{u_i = u_{i-1} + \textrm{simpson}_i(f_{\dot{u}}(t, y, x, \lambda, p))}  &&&& i = 1, \ldots, n \\
         & \mathrlap{\bar{z}_{\textrm{ex},i} = \textrm{hermite}_i(z_{\textrm{ex}}, f_{\dot{z},\textrm{ex}}(t, y, x, \lambda, p))} &&&& i = 1, \ldots, n \\
         & \mathrlap{z_{\textrm{ex},i} = z_{\textrm{ex},i-1} + \textrm{simpson}_i(f_{\dot{z},\textrm{ex}}(t, y, x, \lambda, p))} &&&& i = 1, \ldots, n \\
         & \mathrlap{\bar{z}_{\textrm{im},i} = \textrm{hermite}_i(z_{\textrm{im}}, \zeta)} &&&& i = 1, \ldots, n \\
         & \mathrlap{z_{\textrm{im},i} = z_{\textrm{im},i-1} + \textrm{simpson}_i(\zeta)} &&&& i = 1, \ldots, n \\
         & \mathrlap{0 = f_{\dot{z},\textrm{im}}(t_i, y_i, \zeta_i, x_i, \lambda_i, p)} &&&& i = 0, \ldots, n \\
         & \mathrlap{\bar{x}_i = (x_{i-1} + x_i)/2} &&&& i = 1, \ldots, n \\
         & \mathrlap{0 = \phi(q_i, p) = \dot{\phi}(q_i, u_i, p) = \ddot{\phi}(t_i, y_i, x_i, \lambda_i, p)}  &&&& i = 0, \ldots, n\\
         & V_{L,k} \leq V_k(t_0, t_f, y_0, y_f, x_{0}, \mathrlap{x_{f}, \lambda_0, \lambda_f, p, S_{b,k}) \leq V_{U,k}} \\
        & \quad\quad \mathrlap{S_{b,k} = \sum_{i=1}^{n} \textrm{simpson}_i(s_{b,k}(t, y, x, \lambda, p))} &&&& k = 1, \ldots, K \\
        & \mathrlap{g_{L} \leq g(t_i, y_i, x_{i}, \lambda_i, p) \leq g_{U}}  &&&& i = 0, \ldots, n\\
         \mbox{with respect to} \quad
         & t_0 \in [t_{0,L}, t_{0,U}] && t_n \in [t_{f,L}, t_{f,U}] \\
         & y_0 \in [y_{0,L}, y_{0,U}] && y_n \in [y_{f,L}, y_{f,U}] \\
         & y_i \in [y_{L}, y_{U}] &&&& i = 1, \ldots, n - 1 \\
         & \bar{y}_i \in [y_{L}, y_{U}] &&&& i = 1, \ldots, n \\
         & \zeta_i \in [\zeta_{L}, \zeta_{U}] &&&& i = 0, \ldots, n \\
         & \bar{\zeta}_i \in [\zeta_{L}, \zeta_{U}] &&&& i = 1, \ldots, n \\
         & x_0 \in [x_{0,L}, x_{0,U}] && x_n \in [x_{f,L}, x_{f,U}] \\
         & x_i \in [x_{L}, x_{U}] &&&& i = 1, \ldots, n - 1 \\
         & \bar{x}_i \in [x_{L}, x_{U}] &&&& i = 1, \ldots, n \\
         & \lambda_i \in [\lambda_L, \lambda_U] &&&& i = 0, \ldots, n \\
         & \bar{\lambda}_i \in [\lambda_L, \lambda_U] &&&& i = 1, \ldots, n \\
         & \gamma_i = 0 &&&& i = 1, \ldots, n \\
         & \bar{\gamma}_i \in [\bar{\gamma}_L, \bar{\gamma}_U] &&&& i = 1, \ldots, n \\
         & p \in [p_{L}, p_{U}].
    \end{aligned}
    \label{hermitesimpsonnlp}
\end{align}
}

\newcommand{\hermitesimpsonkincon}{
    \begin{align}
         0 &= \dot{\phi}(q, u, p) = G(q, p) u,\\
         0 &= \ddot{\phi}(t, y, x, \lambda, p) = G(q, p) \dot{u} + \dot{G}(q, p) u = G(q, p) f_{\dot{u}}(t, y, x, \lambda, p) + \dot{G}(q, p) u.
    \end{align}
}

\newcommand{\hermitesimpsonimplicit}{
\begin{align}
    \begin{aligned}
    \mbox{subject to} \quad
         & \bar{u}_i = \textrm{hermite}_i(u, \upsilon) && i = 1, \ldots, n \\
         & u_i = u_{i-1} + \textrm{simpson}_i(\upsilon)  && i = 1, \ldots, n \\
         & M(q_i, p)\upsilon_i + G(q_i, p)^T \lambda_i =
          f_{\textrm{app},i} -
            f_{\textrm{inertial},i} && i = 0, \ldots, n \\
         & M(\bar{q}_i, p)\bar{\upsilon}_i + G(\bar{q}_i, p)^T \bar{\lambda}_i =
          \bar{f}_{\textrm{app},i} -
            \bar{f}_{\textrm{inertial},i} && i = 1, \ldots, n \\
    \mbox{with respect to} \quad
         & \upsilon_i \in [-\upsilon_{B}, \upsilon_{B}] && i = 0, \ldots, n \\
         & \bar{\upsilon}_i \in [-\upsilon_{B}, \upsilon_{B}] && i = 1, \ldots, n.
    \end{aligned}
\end{align}
}

%% END MACROS SECTION

% PREPRINT ONLY:
\usepackage[charter]{mathdesign}
\usepackage{mathtools}

\begin{document}
\vspace*{0.2in}

% Title must be 250 characters or less.
\begin{flushleft}
{\Large
    \textbf\newline{OpenSim Moco: Musculoskeletal optimal control}
}
    \newline
{\Large
    \textbf\newline{S1 Appendix}
}
\newline
% Insert author names, affiliations and corresponding author email (do not include titles, positions, or degrees).
\\
Christopher L. Dembia\textsuperscript{1\Yinyang},
Nicholas A. Bianco\textsuperscript{1\Yinyang},
Antoine Falisse\textsuperscript{2},
Jennifer L. Hicks\textsuperscript{3},
Scott L. Delp\textsuperscript{1,3,4}
\\
\bigskip
\textbf{1} Department of Mechanical Engineering, Stanford University, Stanford, California, United States of America
\\
\textbf{2} Department of Movement Sciences, KU Leuven, Leuven, Belgium
\\
\textbf{3} Department of Bioengineering, Stanford University, Stanford, California, United States of America
\\
\textbf{4} Department of Orthopaedic Surgery, Stanford University, Stanford, California, United States of America
\\
\bigskip

% Insert additional author notes using the symbols described below. Insert symbol callouts after author names as necessary.
%
% Remove or comment out the author notes below if they aren't used.
%
% Primary Equal Contribution Note
\Yinyang These authors contributed equally to this work.

% Additional Equal Contribution Note
% Also use this double-dagger symbol for special authorship notes, such as senior authorship.
%\ddag These authors also contributed equally to this work.

% Current address notes
%\textcurrency Current Address: Dept/Program/Center, Institution Name, City, State, Country % change symbol to "\textcurrency a" if more than one current address note
% \textcurrency b Insert second current address
% \textcurrency c Insert third current address

% Deceased author note
%\dag Deceased

% Group/Consortium Author Note
%\textpilcrow Membership list can be found in the Acknowledgments section.

\end{flushleft}

\linenumbers

% Use "Eq" instead of "Equation" for equation citations.

\subsection*{Moco's optimal control problem}

Moco poses the optimal control problem shown in Eq~\ref{ocp}. We seek the time-dependent states $y(t)$ and controls $x(t)$ that minimize a sum of costs $J_j$ with weights $w_j$. The states include generalized coordinates $q(t)$, generalized speeds $u(t)$, and auxiliary states $z(t)$, such as muscle activations. We may also seek time-invariant parameters $p$, the initial time of the motion $t_0$, or the final time of the motion $t_f$. We place lower ($L$) and upper ($U$) bounds on $t_0$, $t_f$, $y(t)$, $x(t)$, and $p$. Additionally, we place lower and upper bounds on the initial and final values of the states and controls; this permits solving problems with prescribed initial and final states such as standing from a squat.

\ocp

The problem contains constraints for the system's multibody dynamics (involving the mass matrix $M$; applied forces $f_\mathrm{app}$ from gravity, muscles, etc.; and centripetal and Coriolis terms $f_\mathrm{inertial}$) and any auxiliary dynamics. The auxiliary dynamics may be expressed as explicit ($f_{\dot{z},\mathrm{ex}}$) or implicit ($f_{\dot{z},\mathrm{im}}$) differential equations; the auxiliary state variables $z$ are partitioned according to whether they are integrated using explicit differential equations ($z_{\mathrm{ex}}$) or implicit differential equations ($z_{\mathrm{im}}$). The system may contain position-level (holonomic) kinematic constraints to, for example, weld a foot to a bicycle pedal. Each constraint is enforced by forces exerted by tissue, bones, bodies, or other parts of the modeled system. These generalized constraint forces are applied in the constrained directions (e.g., the six degrees of freedom between the foot and pedal), and we introduce time-varying Lagrange multiplier variables $\lambda$ to solve for these forces. The derivatives of the kinematic constraints $\phi$ with respect to the generalized coordinates yields the kinematic constraint Jacobian $G$; the transpose of this matrix converts the Lagrange multipliers into generalized forces along the system's degrees of freedom. See below for details on how Moco handles kinematic constraints.

Additionally, problem contains boundary constraints $V_k$ (with bounds $V_{L,k}$ and $V_{U,k}$) and algebraic path constraints $g$ over the motion (with time-invariant bounds $g_L$ and $g_U$). The cost terms and boundary constraints may depend on initial and final time; states; controls; kinematic constraint multipliers (required for joint reactions); time-invariant parameters; and an integral, $S_{c,j}$ or $S_{b,k}$, over the motion.

\subsection*{Kinematic constraints}

Support for kinematic constraints is a key feature of Moco. Understanding how kinematic constraints are handled in multibody dynamics is valuable for understanding how Moco handles such problems. Consider a two-dimensional point mass system with coordinates $q_x$ and $q_y$ constrained to a parabola $0 = q_y - q_x^2$. To prevent the point mass from violating the constraint, we must apply a force perpendicular to the parabola. Each constraint has a corresponding scalar force variable, called a Lagrange multiplier $\lambda$. We must solve for the required magnitudes of these constraint forces, but the direction in which we apply each of these forces is determined by the derivative of the constraint equations. We gather the derivatives of the constraint equations in the kinematic constraint Jacobian matrix $G$. Each row in this matrix contains the derivative of a single constraint equation with respect to each degree of freedom, and the matrix has a column for each degree of freedom. For the parabola example, the Jacobian is $( -2q_x, 1)$. The transpose of this matrix, $G^T$, contains columns that are vectors in the state space which are perpendicular to each constraint. For our single constraint, the vector $(-2q_x, 1)^T$ is perpendicular to the parabola. The matrix-vector product between the Jacobian transpose and the Lagrange multipliers, $G^T \lambda$, yields the vector of generalized forces (whose length is the number of degrees of freedom) necessary for enforcing the kinematic constraints. For our point mass example, the generalized forces yielded by the vector-scalar product  $(-2q_x, 1)^T \lambda$ keep the point mass on the parabola. To apply these forces to the multibody system, we include the $G^T \lambda$ term in the multibody dynamics equations of motion.

When simulating a multibody system with time-stepping forward integration, we first ensure the initial generalized coordinates and speeds satisfy the kinematic constraints $\phi(q) = 0$ and their first derivative $\dot{\phi}(q, u) = 0$ via a root-solve. During the integration, we solve for generalized accelerations and Lagrange multipliers that obey the multibody dynamics equations of motion and the second derivative of the kinematic constraints, $\ddot{\phi}(q, u, \dot{u}) = 0$. Numerically integrating the resulting generalized accelerations yields generalized coordinates and speeds that approximately lie on the constraint manifold defined by $\phi(q) = 0$; to fix any errors in the constraints caused by numerical integration error, we project the generalized coordinates and speeds back onto the constraint manifold~\cite{Sherman:2011byc}.

In direct collocation, we solve for the entire trajectory of the system—including the generalized coordinates, generalized speeds, and Lagrange multipliers—all at once. When expressing multibody dynamics as implicit differential equations, the generalized accelerations are also unknowns. How we solve for the trajectory of the system in the presence of kinematic constraints depends on the transcription scheme; see the remainder of this Appendix for details.

\subsection*{Moco's optimal control problem with prescribed kinematics}

A common task in musculoskeletal biomechanics is to estimate the muscle and actuator behavior that drove an observed motion. We can solve this problem by minimizing the error between the observed motion and the simulated motion, as with Computed Muscle Control (using the ``slow target'')~\cite{Thelen:2003bba} or \textit{MocoTrack}. Alternatively, we can prescribe the motion exactly, as with Static Optimization~\cite{Crowninshield:1981}, electromyography-driven simulation~\cite{Lloyd:2003}, and the Muscle Redundancy Solver~\cite{Groote:2016dq}. Consider a two-dimensional point mass with coordinates $q_x$ and $q_y$ for which we prescribe a circular motion via the functions $\hat{q}_x(t) = \cos(t)$ and $\hat{q}_y(t) = \sin(t)$. We can either add these functions to the kinematic constraints $\phi(q)$, or we can substitute these functions into the equations of motion, thereby eliminating the variables $q_x$ and $q_y$. With Moco, users can choose either the former approach through OpenSim's \textit{Coordinate}, or the latter (and usually preferable) approach using \textit{PositionMotion}, a new component that employs Simbody's \textit{Motion} class. Prescribing kinematics by eliminating variables leads to a problem that is robust and fast---the nonlinear multibody dynamics are removed from the optimization problem---but prevents predicting kinematic deviations from the observed motion.

When we prescribe kinematics in Moco by eliminating variables, we replace the problem in Eq~\ref{ocp} with the following:

\prescribed

We replace the kinematic variables $q$ and $u$ with known quantities $\hat{q}$ and $\hat{u}$. The system still depends on auxiliary state variables $z$ and control variables $x$, and includes auxiliary dynamics. If none of the parameter variables affect the multibody system, then the multibody dynamics are reduced to a force balance: muscles and other force elements must generate the net generalized forces determined by the kinematics and external loads data.

Whether the motion is prescribed by adding constraints or eliminating variables, OpenSim supplements the modeled force elements with Lagrange multipliers to ensure the prescribed motion is achieved. When using \textit{PositionMotion} with Moco, we require that the prescribed motion's Lagrange multipliers are zero, thereby ensuring the motion is fully generated by the modeled force elements. The easiest way to prescribe kinematics in Moco is to use the \textit{MocoInverse} tool, which uses \textit{PositionMotion} internally.

\subsection*{Transcription schemes}

\subsubsection*{Trapezoidal transcription}

The trapezoidal scheme transcribes the optimal control problem into a nonlinear program by approximating integrals using the trapezoidal rule. As a second-order scheme, trapezoidal transcription exhibits accuracy that improves four-fold when halving the mesh interval (i.e., time step).

We discretize the continuous variables $t$, $y$, $x$, and $\lambda$ on a mesh of time points $t_i$ defined by dimensionless time $\tau_i$, yielding $n$ mesh intervals with durations $h_i$:

\traptau

For conciseness, we define the following function:

\trapfunc

where $\mathrm{trap}_i(F(\eta, p)))$ is a trapezoidal rule approximation of the area under the function $F$ for mesh interval $i$, and $\eta$ represents any subset of continuous variables. We define the explicit multibody dynamics function as:

\explicitmultibody

The mass matrix $M$, the centripetal and Coriolis forces $f_\mathrm{inertial}$, and kinematic constraint Jacobian $G$ are computed by Simbody (OpenSim's multibody dynamics engine) using order-N recursive algorithms\footnote{In contrast to software based on hand-written or symbolically-derived equations of motion, Simbody does not compute the complete mass matrix explicitly.}~\cite{Sherman:2011byc}. The applied forces $f_\mathrm{app}$ can include any force elements supported by Simbody, and are often defined by OpenSim components that provide access to existing Simbody force elements (e.g.,  \textit{SmoothSphereHalfSpaceForce}) or that define custom Simbody force elements (e.g., \textit{DeGrooteFregly2016Muscle}). Simbody computes the constraint Jacobian $G$ based on any Simbody kinematic constraints that the OpenSim model adds to the system. In Moco, the Lagrange multipliers $\lambda$ are explicit optimization variables; this approach differs from that in time-stepping forward integrations, in which Simbody solves for generalized accelerations and Lagrange multipliers simultaneously.

The result of the trapezoidal transcription, with multibody dynamics expressed as explicit differential equations, is the following nonlinear program:

\trapnlp

In this form, the problem can be solved directly by a nonlinear program solver. We introduce the algebraic (control) variable $\zeta$ as the derivative of auxiliary state variables whose dynamics are expressed with an implicit differential equation.

When expressing the multibody dynamics implicitly, we remove the constraint involving $f_{\dot{u}}$, introduce generalized accelerations as an algebraic variable $\upsilon$ and enforce multibody dynamics in “inverse dynamics” form:

\trapimplicit

The constant $\upsilon_B$ is a large positive number (1000 by default).

The dynamic, kinematic, and path constraints are enforced at a set of discrete time points, so the quadratic spline approximation to the continuous variables may violate the original continuous-time constraints between the discrete time points. For this reason, a mesh with more points leads to a more accurate solution.

Our implementation of trapezoidal transcription handles kinematic constraints, but not in the most robust fashion. We enforce $\phi$ but not its time derivatives; enforcing the constraints at only the position level yields an index-3 system of differential-algebraic equations, which are challenging to solve~\cite{Hairer:1996,Campbell:2016,Betts:2010} (the differential-algebraic equations are ``index-3'' because the algebraic constraints $\phi$ must be differentiated three times to convert the system of differential-algebraic equations into ordinary differential equations). To improve numerical conditioning, we minimize the Lagrange multipliers associated with the constraints (with weight $w_\lambda$; see Eq~\ref{trapnlp}).

When kinematics are prescribed (see “Moco's optimal control problem with prescribed kinematics”), multibody dynamics must be expressed implicitly and kinematic constraints are not enforced; we expect the prescribed kinematics to already obey the constraints.

\subsubsection*{Hermite-Simpson transcription}

The Hermite-Simpson scheme transcribes the optimal control problem into a nonlinear program by approximating integrals using a Hermite interpolant and Simpson integration rule. As a third-order scheme, Hermite-Simpson transcription exhibits accuracy that improves eight-fold when halving the mesh interval.

We use a similar dimensionless time mesh as for the trapezoidal scheme, with $n$ mesh intervals with durations $h_i$. We also introduce collocation points at the midpoints of the mesh intervals, leading to a total of $2n + 1$ time points at which we discretize the continuous variables:

\hermitesimpsontau

where $ \bar{\tau}_i $ denote mesh interval midpoints. For conciseness, we define the following functions:

\hermitesimpsonfuncs

where $\mathrm{hermite}_i()$ represents the Hermite interpolant, which yields the value of the midpoint of a segment in a Hermite spline, and $\mathrm{simpson}_i()$ represents the Simpson integration rule. Again, $F$ is a function for mesh interval $i$, and $\eta$ represents any subset of continuous variables.

Using the explicit multibody dynamics function $f_{\dot{u}}$ defined previously, Hermite-Simpson transcription results in the following nonlinear program:

\hermitesimpsonnlp

The variables $\bar{y}_i$ (including $\bar{q}_i$, $\bar{u}_i$, $\bar{z}_{\mathrm{ex},i}$, $\bar{z}_{\mathrm{im},i}$), $\bar{x}_i$, $\bar{\zeta}_i$, $\bar{\lambda}_i$, and $\bar{\gamma}_i$ are associated with the midpoint of mesh interval $i$.

Eq~\ref{hermitesimpsonnlp} includes the first and second time derivatives of the kinematic constraints:

\hermitesimpsonkincon

\begin{equation}
\dot{q}(t) = f_{\dot{u}}(q, u, \gamma, p) = u + G(q, p)^T \gamma
\end{equation}

Including these constraints reduces the index of the differential-algebraic equations from 3 to 1, which provides numerical benefits~\cite{Hairer:1996,Campbell:2016,Betts:2010}. However, simply appending the kinematic constraints to the unconstrained optimal control problem would overconstrain the resulting non-linear program, so we use the method for handling kinematic constraints that was introduced by Posa and colleagues~\cite{Posa:2016}. In this method, Lagrange multipliers, $\lambda$, are introduced as continuous variables, rather than computed by Simbody as in a typical time-stepping forward integration. In addition, we introduce velocity correction variables, $\gamma$, whose multiplication with the transpose of the kinematic constraint Jacobian, $G(q, p)^T \gamma$, yields a correction to the generalized speeds that is perpendicular to the constraint manifold $\phi = 0$. This correction is necessary because the Hermite splines cannot simultaneously satisfy both the differential equations at the mesh interval midpoints and the kinematic constraints at the mesh points. Since kinematic constraints are already enforced at the mesh points, this correction term only appears at the mesh interval midpoints (i.e., $G(\bar{q}_i, p)^T \bar{\gamma}_i$), and velocity corrections at the mesh points are simply zero (i.e., $\gamma_i = 0$ in Equation 11). Although Simbody supports non-holonomic and acceleration-level constraints, Moco only supports holonomic (position-level) constraints.
% TODO (We are planning to enforce algebraic constraints at the midpoints). Algebraic constraints are not enforced at the midpoints of the mesh intervals, but exhibit fourth-order accuracy at these points~\cite{Posa:2016}.

For implicit multibody dynamics, we again remove the constraints involving $f_{\dot{u}}$ and introduce generalized accelerations as algebraic variables $\upsilon$ to enforce multibody dynamics in “inverse dynamics” form:

\hermitesimpsonimplicit

\subsection*{Moco's direct collocation solvers}

Moco provides two solvers as subclasses of \textit{MocoSolver}: \textit{MocoCasADiSolver} uses the third-party CasADi library~\cite{Andersson:2019}, and \textit{MocoTropterSolver} uses a direct collocation solver we developed named Tropter. CasADi is an open-source package for algorithmic differentiation and is a bridge to nonlinear program solvers IPOPT~\cite{Wachter:2006}, SNOPT~\cite{Gill:2005}, and others.

Gradient-based nonlinear program solvers require the gradient of the objective, the Jacobian of the constraints, and sometimes the Hessian of the Lagrangian~\cite{Betts:2010}. To maximize computational efficiency, these derivatives are ideally computed exactly through either analytic expressions or algorithmic differentiation~\cite{Andersson:2019,Walther:2003}. OpenSim's main distribution does not provide exact derivatives, so we use finite differences. CasADi is an ideal library for employing direct collocation, but two limitations led us to create Tropter: CasADi did not initially support finite differences, and CasADi's open-source license is more restrictive than OpenSim's. More recent versions of CasADi support finite differences and CasADi understands the structure of the nonlinear program objective and constraint functions, allowing for potentially more efficient finite difference calculations than with Tropter, which treats the nonlinear program objective and constraints as black-box functions~\cite{Patterson:2012}. If OpenSim provides exact derivatives in the future, we can exploit the algorithmic differentiation modes in Tropter and CasADi~\cite{Falisse:2019a}. Those distributing Moco as a dependency of closed-source software may prefer distributing Moco without CasADi, as CasADi's ``weak copyleft'' GNU Lesser General Public License 3.0 places requirements on how CasADi is redistributed.

\nolinenumbers

% Either type in your references using
% \begin{thebibliography}{}
% \bibitem{}
% Text
% \end{thebibliography}
%
% or
%
% Compile your BiBTeX database using our plos2015.bst
% style file and paste the contents of your .bbl file
% here. See http://journals.plos.org/plosone/s/latex for
% step-by-step instructions.
%

\bibliography{MocoPaper}

\end{document}


